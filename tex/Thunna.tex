% !TEX TS-program = xelatex
% !TEX encoding = UTF-8

% This is a simple template for a XeLaTeX document using the "article" class,
% with the fontspec package to easily select fonts.

\documentclass[11pt]{article} % use larger type; default would be 10pt

\usepackage{fontspec} % Font selection for XeLaTeX; see fontspec.pdf for documentation
\defaultfontfeatures{Mapping=tex-text} % to support TeX conventions like ``---''
\usepackage{xunicode} % Unicode support for LaTeX character names (accents, European chars, etc)
\usepackage{xltxtra} % Extra customizations for XeLaTeX
\usepackage{multirow}
\usepackage{amssymb}
\usepackage{diagbox}
\usepackage{textcomp}
\usepackage{adjustbox}
\usepackage{gb4e}
\noautomath
\let\eachwordone=\it
\newcommand{\comment}[1]{}

%\setmainfont{Charis SIL} % set the main body font (\textrm), assumes Charis SIL is installed
%\addfontfeature{Uppercase Eng alternates=Large eng on baseline}
\setmainfont{Charis SIL}[
%Renderer=Graphite,
RawFeature={Uppercase Eng alternates=Capital N with tail},
RawFeature={Rams horn alternates=Large bowl},
RawFeature={Numbers=OldStyle},
SmallCapsFeatures={Letters=SmallCaps}]
%\setsansfont{Deja Vu Sans}
%\setmonofont{Deja Vu Mono}

% other LaTeX packages.....
\usepackage{geometry} % See geometry.pdf to learn the layout options. There are lots.
\geometry{a4paper} % or letterpaper (US) or a5paper or....
%\usepackage[parfill]{parskip} % Activate to begin paragraphs with an empty line rather than an indent

\usepackage{graphicx} % support the \includegraphics command and options

\title{A Grammar Of Thunná}
\author{ultlang}
%\date{} % Activate to display a given date or no date (if empty),
		 % otherwise the current date is printed 
\begin{document}
\maketitle

\section{Phonology}
	\subsection{Phonological inventory}
		\begin{center}
		\begin{adjustbox}{width={\textwidth},totalheight={\textheight},keepaspectratio}%
		\renewcommand{\arraystretch}{1.25}
			\begin{tabular}{|c|c||c|c|c|c|c|c|}
				
				\hline
				\multicolumn{2}{|c||}{\multirow{2}{*}{Consonants}} & \multirow{2}{*}{Bilabial} & \multicolumn{2}{c|}{Alveolar} & \multirow{2}{*}{Palatal}   & \multirow{2}{*}{Velar} & \multirow{2}{*}{Uvular} \\
				\cline{4-5}
				\multicolumn{2}{|c||}{}                            &                           & Non-sibilant    & Sibilant    &                            &                        &                         \\ \hline\hline
				 
				\multicolumn{2}{|c||}{Nasal}                       & m                         & \multicolumn{2}{c|}{n}        & ɲ <ň>                      & ŋ                      &                         \\ \hline
				
				\multirow{4}{*}{Stop / Affricate} & Tenuis         & p                         & t               & t͡s <c>     & c\textasciitilde{}t͡ʃ <č>    &  k                   & q                       \\ \cline{2-8}
												  & Aspirate       & pʰ <ph>                   & tʰ <th>         & t͡sʰ <ch>   & cʰ\textasciitilde{}t͡ʃʰ <čh> & kʰ <kh>              & qʰ <qh>                 \\ \cline{2-8}
												  & Voiced         & b                         & d               & d͡z <ʒ>     & ɟ\textasciitilde{}d͡ʒ <ǯ>    &  g                   &                         \\ \cline{2-8}
												  & Implosive      & ɓ <bh>                    & ɗ <dh>          &             &                              &                      &                         \\ \hline
				
				\multirow{3}{*}{Fricative} & Tenuis                & ɸ <f>                     & ɬ <ś>           & s           & ʃ <š>                      & x\textasciitilde{}h <x> &                        \\ \cline{2-8}
										   & Aspirate              &                           &                 & sʰ <sh>     & ʃʰ <šh>                    &                         &                        \\ \cline{2-8}
										   & Voiced                & β <v>                     & ɮ <ź>           & z           & ʒ <ž>                      &                         &                        \\ \hline
										   
				\multirow{2}{*}{Approximant} & Plain               & w                         & \multicolumn{2}{c|}{r}        & j <y>                      & ɰ <h>                  &                         \\ \cline{2-8}
											 & Lateral             &                           & \multicolumn{2}{c|}{l}        &                            &                        &                         \\ \hline				
			\end{tabular}
			\end{adjustbox}
			
			\medskip
					
			\begin{tabular}{|c||c|c|c|}
				\hline
				\textbf{Vowels} & Front & Central & Back  \\ \hline\hline
				Close           & i     &         & ɯ <u> \\ \hline
				Mid             & e     & ə       & o     \\ \hline 
				Open            &       & a       &       \\ \hline
			\end{tabular}
		
		\end{center}

	All vowels can additionally be long, which is indicated with an acute accent in the orthography (/aː/ <á>).

	\subsection{Phonotactics}
	The Thunna syllable follows the structure of (C₁)V(C₂), where C₁ can be any consonant and C₂ prohibits voiced and implosive stops and aspirated consonants.\\
	Additionally, an aspirated and unaspirated plosive cannot border each other -- in the orthography, f.e. <apqhi> actually represents /apʰqʰi/ rather than /apqʰi/.

	\subsection{Allophony}
	The tenuis stops are realised as /ʔ/ in coda position.


\section{Morphology}

\subsection{Verbs}

Thunná marks verbs both for subject and for the direct object.

\begin{exe}
  \ex
  \gll Iʒačha  -p.\\
	   die     \textsc{1} \\
  \glt ``I die.''
  
  \ex
  \gll Iʒačha  -ph -í -qhu.\\
	   die     -\textsc{1 ›} \textsc{3} -\textsc{caus} \\
  \glt ``I kill them.''
\end{exe}

For transitive verbs, the intransitive suffix doesn't signify the absence or indefiniteness of the object, but rather reflexivity (or reciprocality for paucal and plural, which are indicated with particles):

\begin{exe}    
  \ex  {
  \gll Iʒačha  -pə -qhu.\\
	   die     -\textsc{1} -\textsc{caus} \\
  \glt ``I kill myself.''}
\end{exe}

If the object is unknown or indefinite, the fourth person needs to be used.
  
\begin{exe}  
  \ex
  \gll Iʒačha  -ph -ó -qhu.\\
	   die     -\textsc{1 ›} \textsc{4} -\textsc{caus} \\
  \glt ``I kill.''
\end{exe}

When an indirect object is present, the person marking must always be of the transitive variety, even if it is unclear (in which case, again, the fourth person is used.)
	
\begin{exe}  
  \ex
  \gll Áźi  -p.\\
	   speak     -\textsc{1}\\
  \glt ``I speak.'', ``I an talking.''
  
  \ex
  \gll Áźi  -ph -í.\\
	   speak     -\textsc{1 ›} \textsc{3} \\
  \glt ``I speak about them.''
  
	\ex
  \gll Áźi  -ph -ó Thunná =xu.\\
	   speak     -\textsc{1 ›} \textsc{4} Thunná with\\
  \glt ``I speak (using) Thunná.''
\end{exe}

In both intransitive and transitive sentences, the tense of a verb is encoded in the subject suffix. A table of all the suffixes follows:

\begin{center}
	\begin{tabular}{|c||c|c|c|c|}
				\hline
				\textbf{Intransitive} & Distant past & Near past   & Present & Future  \\ \hline\hline
				\textsc{1}            & -sp          & -m          & -p      & -f      \\ \hline
				\textsc{2}            & -ht          & -n          & -t      & -s      \\ \hline
				\textsc{3}            & -c           & -ň          & -č      & -š      \\ \hline
				\textsc{4}            & -q           & -ŋ          & -ś      & -ź      \\ \hline
	\end{tabular} 
\end{center}

\begin{center}
	\begin{tabular}{|c||c|c|c|c|c|c|c|c|}
				\hline
				\textbf{Transitive} & \textsc{pdis.a} & \textsc{pnea.a} & \textsc{pre.a} & \textsc{fut.a} & \textsc{p.sg}         & \textsc{p.du} & \textsc{p.pl}\\ \hline\hline
				\textsc{1}          & -v-             & -pp-            & -ph-           & -f-            & -oə\textasciitilde{}ə & -o            & -oa          \\ \hline
				\textsc{2}          & -sh-            & -tt-            & -th-           & -s-            & -u                    & -ú            & -ú           \\ \hline
				\textsc{3}          & -šh-            & -čč-            & -čh-           & -š-            & -e                    & -i            & -í           \\ \hline
				\textsc{4}          & -qh-            & -w-             & -x-            & -q-            & \multicolumn{3}{c|}{-ó}                              \\ \hline
	\end{tabular} 
\end{center}

\begin{exe}  
  %\exi{(1)}
  \ex
  \gll Asqali  -tt                -oə          aw?        \\
	   see     -\textsc{pnea.2 ›} \textsc{1sg} \textsc{q} \\
  \glt ``Did you see me?''
  
  %\exi{(2)}
  \ex
  \gll Iʒačha  -s.              \\
	   die     -\textsc{fut.2}  \\
  \glt ``You're going to die.''
	
  %\exi{(3)}
  \ex
  \gll Áźi   -šh                -ó.           \\ %qhae  n.           \\
	   speak -\textsc{pdis.4 ›} \textsc{3sg}  \\ %here  \textsc{loc} \\
  \glt ``It has been said.''
\end{exe}

\subsection{Nouns}

Most of the morphology of nouns is accomplished using postpositional particles or clitics, such as \textit{n} (\textsc{loc}) or \textit{xu} (\textsc{instr}).\\
Thunna is split-ergative - it uses \textsc{nom-acc} alignment in the present and future and \textsc{erg-abs} in both past tenses. These are also marked with particles, which inflect for number.
\begin{exe}  
  \ex
  \gll Asqali  -čč                -e           ňiʒe naŋaq  =tá.            \\   %-aq some sort of derivation?
	   see     -\textsc{pnea.3 ›} \textsc{3sg} cat  worker \textsc{erg.pl} \\
  \glt ``The workers saw a cat.''
  
    \ex
  \gll Asqali  -š                 -e           ňiʒe =hi             naŋaq  =min.            \\            %-aq some sort of derivation?
	   see     -\textsc{fut.3 ›}  \textsc{3sg} cat  \textsc{acc.sg} worker \textsc{nom.pl} \\
  \glt ``The workers will see a cat.''
\end{exe}

\begin{center}
	\begin{tabular}{|c||c|c|c|c|}
				\hline
				\textbf{Particles} & \textsc{erg} & \textsc{abs} & \textsc{nom} & \textsc{acc} \\ \hline\hline
				\textsc{sg}        & tha          & śi           & mu           & hi           \\ \hline
				\textsc{du}        & dhá          & źí           & maə          & hoa          \\ \hline
				\textsc{pl}        & tá           & śon          & min          & hin          \\ \hline 
	\end{tabular} 
\end{center}

The singular absolutive and accusative particles are often left out.

Most other particles don't have separate singular, dual and plural forms, so the additional particles \textit{qa} (\textsc{du}) and \textit{en} (\textsc{pl}) are used after the particle.

\begin{exe}  
  \ex
  \gll Xoyi  -w                 -u           dhámi =xu            =en.        \\
	   give  -\textsc{pnea.4 ›} \textsc{2sg} spear \textsc{instr} \textsc{pl} \\
  \glt ``You were given some spears.''
\end{exe}

%\section{}

%\subsection{}



\end{document}
